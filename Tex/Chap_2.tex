\chapter{相关技术}

本章介绍与本文研究相关的关键技术基础,包括大语言模型、深度强化学习、Web应用防火墙、入侵与攻击模拟、生成式对抗样本以及智能验证方法。这些技术构成了基于LLM的BAS系统的理论支撑和技术基础。

\section{大语言模型基础}

大语言模型(Large Language Model, LLM)是基于深度神经网络架构的自然语言处理模型,通过在海量文本语料上进行预训练,学习语言的统计规律和语义知识,从而具备强大的文本理解与生成能力\cite{llm_survey_2023}。近年来,随着Transformer架构\cite{transformer_2017}和规模化训练技术的发展,LLM在代码生成、逻辑推理、任务规划等方面展现出超越传统方法的性能。

\subsection{Transformer架构与自回归生成}

现代LLM普遍采用Transformer架构作为核心,该架构基于自注意力机制(Self-Attention Mechanism)实现序列建模。给定输入序列 $\mathbf{x} = (x_1, x_2, \ldots, x_n)$,Transformer通过多层编码器或解码器,将输入映射到高维语义空间,捕获长距离依赖关系。

在生成任务中,LLM采用自回归(Autoregressive)方式逐token生成输出序列。具体而言,模型在每个时间步 $t$ 根据已生成的前缀序列 $(y_1, \ldots, y_{t-1})$ 预测下一个token $y_t$ 的概率分布:
\begin{equation}
P(y_t | y_{1:t-1}, \mathbf{x}) = \text{softmax}(\mathbf{W}_o \cdot \mathbf{h}_t)
\end{equation}
其中,$\mathbf{h}_t$ 是第 $t$ 步的隐藏状态向量,$\mathbf{W}_o$ 是输出层权重矩阵。通过链式法则,整个序列的生成概率可表示为:
\begin{equation}
P(\mathbf{y} | \mathbf{x}) = \prod_{t=1}^{T} P(y_t | y_{1:t-1}, \mathbf{x})
\end{equation}

这种逐token生成的方式使得LLM能够灵活地生成任意长度的序列,且能够通过条件输入(如攻击类型、目标环境描述)实现可控生成,这为攻击载荷的动态生成提供了技术基础。

\subsection{预训练与微调范式}

LLM的训练通常分为预训练(Pre-training)和微调(Fine-tuning)两个阶段:

\textbf{预训练阶段}:模型在大规模无标注文本语料(如网页、书籍、代码库)上进行自监督学习。常用的预训练目标包括:
\begin{itemize}
\item \textbf{因果语言建模(Causal Language Modeling, CLM)}:预测序列中下一个token,目标函数为:
\begin{equation}
\mathcal{L}_{\text{CLM}} = -\sum_{t=1}^{T} \log P(x_t | x_{1:t-1}; \theta)
\end{equation}
\item \textbf{掩码语言建模(Masked Language Modeling, MLM)}:随机掩盖序列中的部分token,训练模型预测被掩盖的内容,常用于双向编码器模型(如BERT)。
\end{itemize}

通过预训练,模型获得了对自然语言和代码语法的通用理解能力,能够处理多种下游任务。

\textbf{微调阶段}:针对特定任务,使用标注数据对预训练模型进行有监督学习。在攻击载荷生成场景中,微调数据通常包括:
\begin{itemize}
\item 历史成功绕过的攻击样本
\item 特定攻击类型的语法规则与变形模式
\item 目标WAF的拦截日志与绕过策略
\end{itemize}

微调过程通常采用标准的交叉熵损失函数,并可引入正则化技术(如权重衰减、Dropout)防止过拟合。此外,近年来提出的参数高效微调方法(Parameter-Efficient Fine-Tuning, PEFT)如LoRA\cite{lora_2021}和Adapter\cite{adapter_2019},能够在保持预训练权重不变的情况下,仅更新少量可训练参数,降低计算成本并保留模型的泛化能力。

\subsection{采样与解码策略}

生成式模型的输出质量与采样策略密切相关。给定模型预测的概率分布 $P(y_t | y_{1:t-1})$,常用的解码方法包括:

\textbf{(1)贪心解码(Greedy Decoding)}:在每个时间步选择概率最高的token:
\begin{equation}
y_t = \arg\max_{v \in \mathcal{V}} P(v | y_{1:t-1})
\end{equation}
该方法简单高效,但容易陷入重复生成或局部最优。

\textbf{(2)束搜索(Beam Search)}:维护 $k$ 个候选序列(束宽),在每步扩展所有候选并保留总概率最高的 $k$ 个。该方法能够探索更广的搜索空间,但计算成本较高,且对于开放式生成任务可能导致过于保守的输出。

\textbf{(3)温度采样(Temperature Sampling)}:引入温度参数 $\tau$ 调节概率分布的平滑度:
\begin{equation}
P_{\tau}(y_t | y_{1:t-1}) = \frac{\exp(z_t / \tau)}{\sum_{v \in \mathcal{V}} \exp(z_v / \tau)}
\end{equation}
其中,$z_t$ 是模型输出的logit值。较高的温度增加随机性,有利于生成多样化样本;较低的温度使分布更尖锐,趋向于确定性输出。

\textbf{(4)Top-k与Top-p采样}:Top-k采样仅从概率最高的 $k$ 个token中采样,而Top-p(Nucleus Sampling)\cite{nucleus_2019}则动态选择累积概率超过阈值 $p$ 的token集合,兼顾了多样性与合理性。

在攻击载荷生成中,需要在"探索新变体"(多样性)和"保持有效性"(语法正确性)之间取得平衡。实践中常采用温度采样结合Top-p策略,并通过强化学习进一步优化生成分布。

\subsection{提示工程与上下文学习}

大语言模型具备强大的上下文学习(In-Context Learning, ICL)能力,即通过精心设计的提示词(Prompt)和少量示例(Few-shot Examples),无需梯度更新即可完成特定任务\cite{gpt3_2020}。在BAS场景中,提示工程可用于:
\begin{itemize}
\item 指定攻击类型与目标(如"生成一个绕过ModSecurity的SQL注入payload")
\item 提供示例攻击样本作为上下文参考
\item 引导模型进行变形策略分析(如"分析该WAF规则的弱点并生成绕过方案")
\end{itemize}

然而,仅依赖提示工程的生成效果受限于模型的预训练知识,对于需要与特定WAF深度交互和自适应学习的场景,仍需结合微调和强化学习技术。

\section{深度强化学习基础}

深度强化学习(Deep Reinforcement Learning, DRL)是机器学习的一个重要分支,通过智能体(Agent)与环境(Environment)的交互试错,学习最优决策策略以最大化长期累积奖励\cite{drl_survey_2019}。相比于监督学习依赖标注数据,强化学习能够在仅有稀疏奖励信号的情况下进行自主探索,这使其特别适合于攻击载荷生成等难以获取大量有效标注的场景。

\subsection{马尔可夫决策过程}

强化学习问题通常建模为马尔可夫决策过程(Markov Decision Process, MDP),定义为五元组 $\langle \mathcal{S}, \mathcal{A}, P, R, \gamma \rangle$:
\begin{itemize}
\item $\mathcal{S}$:状态空间,表示环境的所有可能状态
\item $\mathcal{A}$:动作空间,表示智能体可采取的所有动作
\item $P: \mathcal{S} \times \mathcal{A} \times \mathcal{S} \to [0, 1]$:状态转移概率函数,$P(s'|s,a)$ 表示在状态 $s$ 执行动作 $a$ 后转移到状态 $s'$ 的概率
\item $R: \mathcal{S} \times \mathcal{A} \to \mathbb{R}$:奖励函数,给出在状态 $s$ 执行动作 $a$ 后获得的即时奖励
\item $\gamma \in [0, 1]$:折扣因子,用于平衡即时奖励与长期奖励
\end{itemize}

智能体的目标是学习一个策略 $\pi: \mathcal{S} \to \mathcal{A}$(或概率策略 $\pi(a|s)$),最大化期望累积奖励:
\begin{equation}
J(\pi) = \mathbb{E}_{\tau \sim \pi} \left[ \sum_{t=0}^{T} \gamma^t R(s_t, a_t) \right]
\end{equation}
其中,$\tau = (s_0, a_0, r_0, s_1, a_1, r_1, \ldots)$ 是轨迹(Trajectory)。

在攻击载荷生成场景中,MDP各要素的映射为:
\begin{itemize}
\item \textbf{状态}:已生成的token序列或其嵌入表示
\item \textbf{动作}:选择下一个token或变异操作
\item \textbf{奖励}:基于WAF反馈(绕过/拦截)和验证结果(有效/无效)计算得出
\item \textbf{状态转移}:由生成模型的自回归过程决定
\end{itemize}

\subsection{价值函数与策略优化}

强化学习的核心是估计状态价值函数和动作价值函数:
\begin{itemize}
\item \textbf{状态价值函数}(State-Value Function):
\begin{equation}
V^\pi(s) = \mathbb{E}_{\pi} \left[ \sum_{t=0}^{\infty} \gamma^t R(s_t, a_t) \mid s_0 = s \right]
\end{equation}
表示从状态 $s$ 开始,遵循策略 $\pi$ 能够获得的期望累积奖励。

\item \textbf{动作价值函数}(Action-Value Function):
\begin{equation}
Q^\pi(s, a) = \mathbb{E}_{\pi} \left[ \sum_{t=0}^{\infty} \gamma^t R(s_t, a_t) \mid s_0 = s, a_0 = a \right]
\end{equation}
表示在状态 $s$ 执行动作 $a$ 后,继续遵循策略 $\pi$ 的期望累积奖励。
\end{itemize}

策略优化方法可分为两大类:

\textbf{(1)基于价值的方法}(Value-Based Methods):如Q-Learning、DQN\cite{dqn_2015},通过学习Q函数近似最优策略:
\begin{equation}
\pi^*(s) = \arg\max_a Q^*(s, a)
\end{equation}

\textbf{(2)基于策略的方法}(Policy-Based Methods):直接优化参数化策略 $\pi_\theta$。策略梯度(Policy Gradient)定理\cite{policy_gradient_2000}给出了梯度计算公式:
\begin{equation}
\nabla_\theta J(\pi_\theta) = \mathbb{E}_{\tau \sim \pi_\theta} \left[ \sum_{t=0}^{T} \nabla_\theta \log \pi_\theta(a_t | s_t) \cdot A^\pi(s_t, a_t) \right]
\end{equation}
其中,$A^\pi(s, a) = Q^\pi(s, a) - V^\pi(s)$ 是优势函数(Advantage Function),用于降低方差。

Actor-Critic方法结合了两者优势,使用Critic网络估计价值函数,Actor网络根据Critic的指导更新策略。

\subsection{近端策略优化(PPO)}

Proximal Policy Optimization(PPO)\cite{ppo_2017}是目前最常用的策略优化算法之一,其核心思想是限制策略更新幅度,避免训练不稳定。PPO引入剪切目标函数:
\begin{equation}
L^{\text{CLIP}}(\theta) = \mathbb{E}_t \left[ \min \left( r_t(\theta) \hat{A}_t, \text{clip}(r_t(\theta), 1-\epsilon, 1+\epsilon) \hat{A}_t \right) \right]
\end{equation}
其中,$r_t(\theta) = \frac{\pi_\theta(a_t | s_t)}{\pi_{\theta_{\text{old}}}(a_t | s_t)}$ 是概率比(Probability Ratio),$\epsilon$ 是剪切阈值(通常取0.1-0.2),$\hat{A}_t$ 是优势估计。

剪切机制确保了新策略不会偏离旧策略太远,从而在提升样本效率的同时保持训练稳定性。PPO因其简单高效且对超参数不敏感,被广泛应用于语言模型的强化学习微调(如RLHF\cite{rlhf_2022})。

\subsection{KL散度约束与奖励建模}

在将强化学习应用于语言生成任务时,常面临两个关键挑战:

\textbf{(1)模式坍塌(Mode Collapse)}:策略优化过程中,模型可能过度利用某些高奖励样本,导致生成多样性下降。为缓解该问题,通常引入KL散度惩罚项:
\begin{equation}
J_{\text{RL}}(\theta) = \mathbb{E}_{x \sim p_{\text{data}}, y \sim \pi_\theta(\cdot|x)} \left[ R(x, y) \right] - \beta \cdot D_{\text{KL}}(\pi_\theta \| \pi_{\text{ref}})
\end{equation}
其中,$\pi_{\text{ref}}$ 是参考策略(通常为监督微调后的模型),$\beta$ 是约束强度。KL散度项限制了策略不会偏离参考策略过远,保持生成质量与合理性。

\textbf{(2)奖励稀疏性(Sparse Reward)}:在攻击场景中,只有成功绕过WAF才能获得正向奖励,导致信号稀疏、训练困难。奖励建模(Reward Modeling)通过训练一个辅助模型预测样本的价值,提供更密集的中间奖励:
\begin{equation}
R_{\text{total}}(x, y) = R_{\text{env}}(x, y) + \alpha \cdot R_{\text{model}}(x, y)
\end{equation}
奖励模型可基于历史成功样本与失败样本训练,学习区分高价值与低价值的生成策略。

\section{Web应用防火墙(WAF)技术}

Web应用防火墙(Web Application Firewall, WAF)是部署在Web应用前端的安全设备或服务,用于检测和阻断恶意HTTP/HTTPS请求,保护后端应用免受SQL注入、XSS、命令注入、路径遍历等攻击\cite{waf_survey_2022}。WAF的核心功能包括请求过滤、威胁检测、日志记录和流量管理。

\subsection{WAF检测机制}

现代WAF通常采用多层检测机制,结合规则引擎、语义分析和机器学习:

\textbf{(1)基于签名的检测}:维护已知攻击模式的规则库,使用正则表达式或字符串匹配识别恶意特征。例如,针对SQL注入的规则可能包括:
\begin{itemize}
\item 检测SQL关键字(如 \texttt{UNION}, \texttt{SELECT}, \texttt{DROP})
\item 识别注释符号(如 \texttt{--}, \texttt{/**/}, \texttt{\#})
\item 匹配特殊字符组合(如 \texttt{' OR '1'='1})
\end{itemize}

这种方法对已知攻击有较高检出率,但存在以下局限:
\begin{itemize}
\item 易被编码绕过(如URL编码、Unicode编码、十六进制表示)
\item 对变形攻击敏感性差(如利用等价语法替换)
\item 规则维护成本高,且滞后于新型攻击手段
\end{itemize}

\textbf{(2)基于语义的检测}:通过解析请求中的SQL语句、JavaScript代码等,分析其语法结构和执行语义。例如,SQL解析器可识别是否存在非预期的SQL操作符或子查询。该方法能够检测到某些绕过技巧,但实现复杂度高,且可能因解析器差异导致误判。

\textbf{(3)基于机器学习的检测}:将HTTP请求特征(如URL长度、参数数量、字符频率、N-gram统计)输入分类模型(如随机森林、SVM、深度神经网络),判断请求是否为攻击\cite{ml_waf_2020}。机器学习WAF的优势在于:
\begin{itemize}
\item 能够学习复杂的特征组合
\item 对未知攻击变体有一定泛化能力
\item 可通过在线学习持续更新
\end{itemize}

然而,机器学习模型也引入了新的脆弱性,即对抗样本攻击。攻击者可通过微调输入特征,使恶意请求被误分类为良性请求,这正是本文研究的核心问题之一。

\subsection{常见开源WAF}

目前主流的开源WAF包括:

\textbf{(1)ModSecurity}:Apache基金会支持的开源WAF,支持灵活的规则配置和多种部署模式(如反向代理、嵌入式模块)。其核心规则集OWASP Core Rule Set(CRS)\cite{owasp_crs}涵盖了常见Web攻击类型,并持续更新以应对新威胁。

\textbf{(2)Naxsi}:基于Nginx的轻量级WAF,采用白名单策略和评分机制。其检测逻辑相对简单,性能开销较低,适合高并发场景。

\textbf{(3)Shadow Daemon}:支持多种Web服务器的模块化WAF,通过数据流分析和异常检测识别攻击。

这些开源WAF为BAS系统的测试与评估提供了可控的实验环境。

\subsection{WAF绕过技术}

攻击者常用的WAF绕过技术包括:
\begin{itemize}
\item \textbf{编码混淆}:使用URL编码、Unicode编码、HTML实体编码等规避规则匹配
\item \textbf{大小写变换}:利用SQL不区分大小写的特性(如 \texttt{SeLeCt})
\item \textbf{注释插入}:在关键字中插入注释符(如 \texttt{SEL/**/ECT})
\item \textbf{等价替换}:使用语义等价的语法结构(如 \texttt{UNION} 替换为 \texttt{UNION ALL})
\item \textbf{协议滥用}:利用HTTP协议特性(如分块传输、多字段提交)绕过检测
\end{itemize}

这些技巧构成了传统Payload字典的基础,而基于LLM的生成方法能够自动组合和创新这些变形策略。

\section{入侵与攻击模拟(BAS)技术}

入侵与攻击模拟(Breach and Attack Simulation, BAS)是一种自动化安全验证技术,通过持续模拟真实攻击场景,评估企业防御体系的有效性\cite{bas_platform_2024}。BAS与传统渗透测试的主要区别在于其自动化、持续性和规模化特征。

\subsection{BAS系统架构}

典型的BAS系统包含以下核心模块:

\textbf{(1)攻击场景库}:预定义的攻击剧本(Playbooks),描述攻击的战术、技术和程序(TTPs)。常见的场景包括:
\begin{itemize}
\item 初始访问(Initial Access):如钓鱼邮件、漏洞利用
\item 权限提升(Privilege Escalation):如本地提权、容器逃逸
\item 横向移动(Lateral Movement):如凭证窃取、内网扫描
\item 数据渗出(Data Exfiltration):如DNS隧道、HTTP隐蔽信道
\end{itemize}

\textbf{(2)攻击执行引擎}:负责解释攻击剧本并生成具体的攻击流量。传统BAS系统通常采用预定义的Payload模板,而本文研究的智能BAS系统则利用LLM动态生成载荷。

\textbf{(3)监控与验证模块}:实时监控攻击执行过程,捕获目标系统的响应,判断攻击是否成功。验证机制包括:
\begin{itemize}
\item 响应码分析(如检测到500错误可能表明SQL注入成功)
\item 响应内容匹配(如页面回显数据库信息)
\item 行为监控(如检测到Shell回连、文件修改)
\end{itemize}

\textbf{(4)结果评估与报告}:汇总测试结果,生成安全态势报告,标识防御薄弱点并提供修复建议。

\subsection{BAS的价值与局限}

BAS的核心价值在于:
\begin{itemize}
\item \textbf{持续验证}:可按需或定期执行,持续评估防御有效性
\item \textbf{成本优势}:自动化执行降低了人工渗透测试的成本
\item \textbf{全面覆盖}:能够覆盖大量攻击场景,避免遗漏
\item \textbf{量化风险}:提供可量化的安全指标,辅助决策
\end{itemize}

然而,传统BAS系统也存在明显局限:
\begin{itemize}
\item \textbf{依赖静态剧本}:攻击手段固定,难以应对快速演进的防御技术
\item \textbf{缺乏适应性}:无法根据目标环境动态调整策略
\item \textbf{验证不足}:仅依赖简单的响应判断,易产生误报
\end{itemize}

本文研究通过引入LLM和强化学习,旨在突破这些局限。

\section{生成式对抗样本技术}

对抗样本(Adversarial Examples)最初在计算机视觉领域被发现,指的是通过微小扰动导致模型误分类的输入\cite{adversarial_2014}。在网络安全领域,对抗样本生成技术被应用于测试WAF、IDS等检测系统的鲁棒性。

\subsection{对抗样本的定义与目标}

在WAF场景中,对抗样本是指同时满足以下条件的攻击载荷 $x'$:
\begin{itemize}
\item \textbf{功能等价性}:$x'$ 与原始攻击载荷 $x$ 在目标系统上具有相同的攻击效果(如触发SQL注入漏洞)
\item \textbf{绕过能力}:$x'$ 能够通过WAF的检测,即 $f_{\text{WAF}}(x') = \text{benign}$,而 $f_{\text{WAF}}(x) = \text{malicious}$
\item \textbf{可用性}:$x'$ 符合目标语言的语法规范,不会因格式错误而被拒绝
\end{itemize}

形式化地,对抗样本生成问题可表述为约束优化:
\begin{equation}
\begin{aligned}
\max_{x'} \quad & P_{\text{WAF}}(x' \text{ is benign}) \\
\text{s.t.} \quad & \text{Effect}(x') = \text{Effect}(x) \\
& \text{Valid}(x') = \text{True} \\
& d(x, x') \leq \delta
\end{aligned}
\end{equation}
其中,$d(\cdot, \cdot)$ 是距离度量(如编辑距离),$\delta$ 是扰动预算。

\subsection{基于搜索的生成方法}

早期的对抗样本生成方法主要基于启发式搜索:

\textbf{(1)遗传算法(Genetic Algorithm, GA)}:将Payload编码为染色体,通过选择、交叉和变异操作演化种群,适应度函数由WAF绕过成功率定义。WAF-A-MoLE\cite{waf_a_mole}即采用该方法。

\textbf{(2)爬山算法(Hill Climbing)}:从初始Payload出发,迭代应用变异操作,保留能够降低WAF检测置信度的变体。

\textbf{(3)模拟退火(Simulated Annealing)}:在爬山算法基础上引入概率性接受劣解,避免陷入局部最优。

这些方法的优势在于实现简单、无需训练数据,但存在搜索效率低、难以处理高维空间等问题。

\subsection{基于学习的生成方法}

近年来,研究者将机器学习技术应用于对抗样本生成:

\textbf{(1)基于序列到序列模型的生成}:使用Seq2Seq模型学习从原始Payload到绕过变体的映射\cite{seq2seq_waf}。训练数据通常由人工标注的变形样本对构成。

\textbf{(2)基于生成对抗网络的生成}:生成器网络生成Payload,判别器网络模拟WAF的检测逻辑,两者对抗训练\cite{gan_waf}。

\textbf{(3)基于强化学习的生成}:将生成过程建模为MDP,通过与真实WAF交互学习最优变异策略。该方法无需大量标注数据,能够适应黑盒环境。

\textbf{(4)基于LLM的生成}:利用预训练语言模型的语义理解能力,通过提示工程或微调生成高质量载荷。结合强化学习,可进一步优化生成策略以适应特定WAF。

\section{智能验证技术}

在BAS系统中,验证环节用于确认生成的攻击载荷是否真正触发了漏洞。仅依赖WAF的放行/拦截结果可能导致误判,因为:
\begin{itemize}
\item WAF放行并不意味着攻击成功(可能Payload无效)
\item WAF拦截也可能误判(False Positive)
\end{itemize}

因此,需要引入更可信的验证机制\cite{verification_2024}。

\subsection{基于响应分析的验证}

通过对比正常请求与攻击请求的响应差异,判断攻击是否生效:

\textbf{(1)响应码验证}:某些攻击会导致异常响应码,如:
\begin{itemize}
\item SQL注入可能触发500内部错误
\item 路径遍历可能返回403禁止访问
\end{itemize}

\textbf{(2)响应内容验证}:检查响应体中是否包含敏感信息,如:
\begin{itemize}
\item 数据库错误信息(如 \texttt{MySQL syntax error})
\item 系统路径泄露(如 \texttt{/etc/passwd})
\item 数据库查询结果(如用户表记录)
\end{itemize}

\textbf{(3)响应时间验证}:针对时间盲注攻击,通过测量响应时间判断Payload是否触发延迟。

这些方法实现简单,但准确性受限于响应特征的明显程度。

\subsection{基于浏览器执行的验证}

对于XSS等需要客户端执行的攻击,单纯分析HTTP响应不足以验证有效性。基于无头浏览器(Headless Browser)的验证方法\cite{browser_verification}通过模拟真实浏览器环境,监控以下行为:

\textbf{(1)DOM变化监测}:检测是否插入了恶意脚本节点或修改了页面结构。

\textbf{(2)JavaScript执行监控}:通过浏览器API(如Chrome DevTools Protocol)捕获脚本执行事件,验证是否触发了预期的恶意操作(如弹窗、Cookie窃取)。

\textbf{(3)网络流量监控}:检测是否有异常的外联请求(如向攻击者服务器发送数据)。

该方法显著降低了误报率,但计算开销较大,需权衡验证准确性与执行效率。

\subsection{基于LLM的语义验证}

LLM可作为智能裁判,对攻击结果进行语义层面的验证。具体而言,将攻击上下文、目标响应和预期效果输入LLM,由模型判断攻击是否成功。例如:
\begin{itemize}
\item 输入:"目标:获取数据库版本;响应:MySQL 5.7.32"
\item 输出:"攻击成功,已获取数据库版本信息"
\end{itemize}

这种方法的优势在于能够处理复杂的上下文关系,但需注意LLM的幻觉问题,必要时应结合规则验证。

\section{本章小结}

本章系统介绍了与本文研究相关的核心技术,包括大语言模型的架构原理、训练范式和生成策略,深度强化学习的理论基础、策略优化算法和KL约束机制,Web应用防火墙的检测机制与绕过技术,入侵与攻击模拟系统的架构与价值,生成式对抗样本的定义与生成方法,以及智能验证技术的多种实现途径。这些技术为后续章节的系统设计、实现与实验评估提供了坚实的理论支撑和技术基础。
