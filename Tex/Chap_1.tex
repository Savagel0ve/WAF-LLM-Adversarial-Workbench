\chapter{绪论}

\section{研究背景}

随着信息技术的快速发展和网络应用的日益普及,网络安全问题已成为影响国家安全、经济发展和社会稳定的重要因素。近年来,网络攻击手段呈现出自动化、智能化、复杂化的演进趋势,传统的被动防御机制正面临前所未有的挑战。据统计,全球网络攻击事件的数量和影响范围逐年增长,其中高级持续性威胁(Advanced Persistent Threat, APT)攻击者所具备的灵活性、创新性和针对性,使得企业和组织的安全防御体系难以有效应对\cite{apt_threat_2024}。

为应对日益严峻的网络安全威胁,主动防御技术逐渐成为网络安全领域的研究热点。入侵与攻击模拟(Breach and Attack Simulation, BAS)作为一种新兴的主动防御技术,旨在通过持续模拟真实攻击者的战术、技术和程序(Tactics, Techniques and Procedures, TTPs),主动发现和验证企业安全防御体系的薄弱环节\cite{bas_platform_2024}。与传统的渗透测试相比,BAS系统能够以更高的频次、更低的成本持续评估安全态势,为企业提供动态的安全风险视图。

然而,现有的BAS系统大多依赖于静态的攻击剧本库(Playbooks)和基于签名的回放机制,存在以下显著局限性:(1)攻击载荷(Payload)更新滞后,难以应对快速演进的Web应用防火墙(Web Application Firewall, WAF)规则;(2)缺乏针对性和自适应能力,无法根据目标环境的特征动态调整攻击策略;(3)验证机制简单,容易产生大量误报(False Positives),降低了测试的有效性和可信度\cite{waf_evasion_survey}。这些问题制约了BAS系统在实战中的应用价值。

近年来,以大语言模型(Large Language Model, LLM)为代表的生成式人工智能技术取得了突破性进展。LLM不仅具备强大的代码生成和语义理解能力,还能够进行复杂的推理和决策,这为BAS系统的智能化转型提供了新的技术路径\cite{llm_security_2024}。特别是,LLM在自然语言处理、程序合成、漏洞分析等领域展现出的能力,使得构建具备"自主思考"能力的攻击代理(Agent)成为可能。同时,深度强化学习(Deep Reinforcement Learning, DRL)技术的成熟,为攻击载荷的对抗性生成提供了有效的优化框架\cite{drl_cybersecurity}。

在此背景下,本研究聚焦于基于LLM的BAS系统攻击载荷自动化生成与智能验证技术,旨在突破传统BAS系统的技术瓶颈,构建一个能够自主生成对抗样本、智能适应防御变化、精确验证攻击效果的下一代BAS系统。研究结合深度强化学习、奖励建模和智能验证技术,探索从"枚举"到"生成"的范式转变,对于提升网络安全主动防御能力具有重要的理论意义和应用价值。

\section{国内外研究现状}

\subsection{对抗样本生成技术}

攻击载荷的生成是BAS系统的核心能力。传统的漏洞扫描器依赖于预定义的Payload字典,容易被WAF基于规则拦截。为了突破这一限制,研究界引入了对抗攻击(Adversarial Attack)的概念,旨在生成能够欺骗检测模型但仍保留攻击语义的样本。

\subsubsection{基于变异模糊测试的WAF逃逸}

早期的研究集中于针对机器学习驱动的WAF进行对抗样本生成。WAF-A-MoLE\cite{waf_a_mole}是该领域的开创性工作,将SQL注入攻击载荷的生成问题建模为针对WAF分类器的黑盒对抗攻击问题。该方法定义了一组语义保留(Semantics-Preserving)的变异算子,如空白字符替换、大小写转换、等价语义替换等,通过遗传算法不断迭代变异后的Payload,直至WAF将恶意Payload误判为良性流量。然而,WAF-A-MoLE主要针对基于机器学习分类器的WAF,其生成的样本在面对基于严格正则规则的传统WAF时效果有限,且变异过程是随机的,缺乏对WAF内部逻辑的深层理解。

\subsubsection{基于深度强化学习的对抗攻击}

针对跨站脚本攻击(Cross-Site Scripting, XSS)的检测模型,研究者引入了深度强化学习技术。相关研究\cite{xss_drl_2025}将Payload生成过程建模为马尔可夫决策过程(Markov Decision Process, MDP),采用Soft Actor-Critic (SAC)算法通过最大化熵来鼓励探索,避免Agent陷入局部最优解,从而生成更加多样化且隐蔽的对抗样本。为解决生成的对抗样本可能因过度变异而失去攻击能力的问题,最新的研究引入了XSS Oracle(通常基于无头浏览器),在训练循环中实时验证生成的Payload是否仍能触发JavaScript执行\cite{xss_oracle_2024}。

\subsubsection{LLM与强化学习结合的生成框架}

随着大语言模型能力的提升,研究者开始探索将LLM的语义理解能力与强化学习的决策优化能力相结合。GPTFuzzer\cite{gptfuzzer_2024}是该方向的代表性工作,提出了生成式黑盒WAF测试方法。其核心思想是让模型按token逐步生成payload,并通过与目标WAF的交互反馈进行强化学习自适应。系统采用两阶段训练策略:先用中等规模payload数据对模型进行预训练,使其具备生成攻击payload的基本能力;再针对具体WAF进行强化学习,将生成分布朝"更可能绕过该WAF"的方向调整。通过奖励建模和KL散度惩罚,该方法在固定请求预算下能发现显著更多的绕过payload。

DEG-WAF\cite{deg_waf}代表了当前该方向的最前沿进展,利用LLM来"编写"具有逃逸能力的Payload。系统包含Payload生成代理、奖励模型、语法采样代理和强化学习代理四个核心组件,采用Proximal Policy Optimization (PPO)算法微调LLM的生成策略。GenSQLi\cite{gensqli}框架则展示了LLM在不需要复杂微调的情况下,仅通过上下文学习(In-Context Learning)即可生成高质量Payload的能力,并提出了"生成-攻击-修复"的自动化闭环。

\subsection{智能验证技术}

在BAS系统中,如何自动化、准确地验证载荷是否成功触发了漏洞,是区分"模拟"与"实战"的关键。

YuraScanner\cite{yurascanner}提出了一种任务驱动(Task-Driven)的扫描与验证方法,利用LLM理解Web页面的语义,自动填充表单、点击按钮,进入传统扫描器无法到达的"深层状态"(Deep States)。系统在无头浏览器中植入传感器,实时监控DOM树的变化,不仅检查HTTP响应中是否包含恶意脚本字符串,还检查浏览器是否真的执行了该脚本,这种基于执行的验证(Execution-based Verification)极大地降低了误报率。

HARM框架\cite{harm}采用LLM-as-a-Judge模式,训练一个专门的LLM来充当"裁判",对目标的回复进行打分。这种机制不仅用于验证攻击是否成功,还被用于强化学习的奖励信号。PwnGPT\cite{pwngpt}在二进制漏洞利用领域展示了反馈修正机制,如果生成的Exploit导致程序崩溃但未获得Shell,验证模块会捕获崩溃时的寄存器状态,计算偏移量,并将这些调试信息反馈给LLM进行修正。

\subsection{研究现状总结}

综合上述研究现状可以看出,基于LLM的BAS系统正处于从实验性研究向工程化应用转化的关键时期。在对抗样本生成方面,已经实现了从基于变异的随机搜索到基于强化学习的自适应生成的演进;在验证技术方面,已经探索了从简单的响应匹配到基于浏览器执行监测的智能验证。

然而,现有研究仍存在以下不足:(1)缺乏系统化的"生成-验证-反馈-评估"工程闭环,大多数研究仅关注单一环节;(2)对攻击载荷的有效性验证不足,容易产生无效样本;(3)在真实WAF环境下的样本效率和绕过率有待进一步提升;(4)缺乏面向BAS场景的系统级设计,难以直接应用于企业安全评估。

\section{本文工作}

针对现有研究的不足,本文开展基于LLM的BAS系统攻击载荷自动化生成与智能验证技术研究。主要研究内容包括:

\subsection{研究内容}

\textbf{(1)基于LLM与强化学习的攻击载荷生成基线构建}

构建端到端的攻击载荷生成流程,包括攻击类型条件输入、逐token生成payload、与目标WAF的黑盒交互、基于反馈的强化学习微调,以及稳定训练所需的KL散度惩罚与PPO更新机制。建立可复现实验基线,为后续系统设计与评估提供方法学基础。

\textbf{(2)面向BAS场景的系统架构设计}

设计并实现"生成-验证-反馈-评估"的完整工程闭环。系统采用前后端分离架构,包括策略模型、参考模型、WAF黑盒环境、奖励建模模块和优化器等核心组件,构建可扩展的BAS系统原型。

\textbf{(3)智能验证机制研究}

将"绕过WAF"与"payload真实有效"解耦,设计并实现验证器(Verifier/Oracle)模块。通过响应差异分析和浏览器执行监测等技术,避免将无效样本当作成功样本回灌训练,提升系统的准确性和可靠性。

\textbf{(4)实验验证与性能评估}

在可复现的开源WAF(如ModSecurity、Naxsi)上进行系统性实验,重点评估样本效率指标(在固定请求预算内发现的绕过payload数量、达到绕过阈值所需的请求数)。通过对照实验和消融实验,验证KL惩罚、奖励建模等关键机制的有效性。

\section{论文组织结构}

本文共分为六章,各章内容安排如下:

\textbf{第一章 绪论}。介绍了本研究的背景与意义,综述了对抗样本生成与智能验证技术的国内外研究现状,阐述了本文的主要研究内容。

\textbf{第二章 相关技术}。介绍大语言模型、深度强化学习、Web应用防火墙、入侵与攻击模拟等相关技术的基本原理,阐述基于LLM与强化学习的生成式测试方法及其关键机制,为后续章节的系统设计和实现奠定理论基础。

\textbf{第三章 基于LLM的BAS系统设计}。详细阐述系统的总体架构设计,包括策略模型、奖励建模、验证器、反馈机制等核心模块的设计思路和实现方案,给出系统的工作流程和关键算法。

\textbf{第四章 系统实现}。介绍系统的具体实现细节,包括开发环境配置、关键模块实现、数据集准备等。

\textbf{第五章 系统实验与评估}。设计并开展系统性实验,对比分析系统在不同WAF、不同攻击类型下的性能表现,通过消融实验验证关键技术组件的有效性。

\textbf{第六章 总结与展望}。总结本文的研究工作和主要结论,分析系统的优势与不足,展望未来的研究方向和改进思路。

\section{本章小结}

本章首先介绍了本文的研究背景与意义,随后综述了对抗样本生成与智能验证技术的国内外研究现状,进一步阐述了本文的研究内容,并给出了论文的组织结构。